%!TEX root = ../main.tex
In this paper we use Bayesian methods to revisit a classic question in empirical finance: which asset pricing factors explain the time-series and cross-section behavior of asset returns? 
To answer this question one must consider an extremely large collection of models formed from all possible subsets of the proposed factors.
And with so many models under consideration, there is a serious risk of overfitting.
We address this challenging high-dimensional model selection problem by calculating and comparing marginal likelihoods in a Bayesian seemingly unrelated regression (SUR) model with multivariate Student-t errors and an objective, training-sample prior.
Unlike approaches based on hypothesis testing, marginal likelihood comparisons automatically penalize models based on complexity, avoiding the problem of over-fitting. 
Our proposed method performs well in a calibrated simulation, selecting the correct model even when we intentionally mis-specify the error distribution.
We then consider an application using monthly returns for ten industry portfolios and twelve leading asset pricing factors.
Our exhaustive search over 49,152 models, $2^{13}$ for each of six error specification suggests that, in addition to the usual suspects, profitability and quality are important for explaining the behavior of returns.

