%!TEX root = main.tex
\section{Application}
\subsection{Data}

\begin{sidewaystable}[ht]
	\centering
	% latex table generated in R 3.2.3 by xtable 1.8-2 package
% Mon Mar 14 16:22:38 2016
\begin{tabular}{ccccccccccc}
  \hline
 & NoDur & Durbl & Manuf & Enrgy & HiTec & Telcm & Shops & Hlth & Utils & Other \\ 
  \hline
Jan 1982 - Dec 2014 & 0.0092 & 0.0071 & 0.0080 & 0.0070 & 0.0074 & 0.0069 & 0.0087 & 0.0087 & 0.0069 & 0.0067 \\ 
  Jan 1982 - Dec 1989 & 0.0156 & 0.0103 & 0.0083 & 0.0066 & 0.0025 & 0.0136 & 0.0130 & 0.0104 & 0.0093 & 0.0081 \\ 
  Jan 1990 - Dec 2014 & 0.0072 & 0.0060 & 0.0079 & 0.0072 & 0.0090 & 0.0047 & 0.0074 & 0.0082 & 0.0061 & 0.0062 \\ 
   \hline
\end{tabular}

	\label{table:avPortfolioReturns}	
	\caption{Average Portfolio Returns}
	\bigskip\bigskip
	% latex table generated in R 3.2.3 by xtable 1.8-2 package
% Mon Mar 14 16:22:38 2016
\begin{tabular}{ccccccccccccc}
  \hline
 & Mkt.RF & HML & SMB & CMA & RMW & ME & IA & ROE & MOM & LIQv & QMJ & HMLDev \\ 
  \hline
Jan 1982 - Dec 2014 & 0.0068 & 0.0031 & 0.0011 & 0.0035 & 0.0037 & 0.0018 & 0.0040 & 0.0055 & 0.0059 & 0.0048 & 0.0046 & 0.0028 \\ 
  Jan 1982 - Dec 1989 & 0.0085 & 0.0059 & -0.0017 & 0.0059 & 0.0041 & -0.0012 & 0.0068 & 0.0071 & 0.0070 & 0.0043 & 0.0060 & 0.0056 \\ 
  Jan 1990 - Dec 2014 & 0.0063 & 0.0022 & 0.0020 & 0.0028 & 0.0036 & 0.0028 & 0.0031 & 0.0050 & 0.0055 & 0.0049 & 0.0041 & 0.0020 \\ 
   \hline
\end{tabular}

	\caption{Average Factor Returns}
	\label{table:avFactorReturns}	
\end{sidewaystable}

\begin{sidewaystable}[ht]
	\centering
	% latex table generated in R 3.2.3 by xtable 1.8-2 package
% Wed Feb 24 16:45:32 2016
\begin{tabular}{ccccccccccccc}
  \hline
 & LIQv & MOM & Mkt.RF & SMB & HML & RMW & CMA & QMJ & ME & IA & ROE & HMLDev \\ 
  \hline
LIQv &  &  &  &  &  &  &  &  &  &  &  &  \\ 
  MOM & -0.02 &  &  &  &  &  &  &  &  &  &  &  \\ 
  Mkt.RF & -0.06 & -0.13 &  &  &  &  &  &  &  &  &  &  \\ 
  SMB & -0.03 & -0.05 & 0.28 &  &  &  &  &  &  &  &  &  \\ 
  HML & 0.03 & -0.15 & -0.32 & -0.13 &  &  &  &  &  &  &  &  \\ 
  RMW & 0.03 & 0.1 & -0.21 & -0.38 & 0.11 &  &  &  &  &  &  &  \\ 
  CMA & 0.02 & 0.02 & -0.4 & -0.09 & 0.71 & -0.06 &  &  &  &  &  &  \\ 
  QMJ & 0.04 & 0.25 & -0.53 & -0.52 & 0.02 & 0.76 & 0.08 &  &  &  &  &  \\ 
  ME & -0.04 & -0.02 & 0.26 & 0.97 & -0.08 & -0.37 & -0.05 & -0.5 &  &  &  &  \\ 
  IA & 0.02 & 0.04 & -0.39 & -0.19 & 0.69 & 0.06 & 0.91 & 0.15 & -0.15 &  &  &  \\ 
  ROE & -0.06 & 0.5 & -0.2 & -0.39 & -0.1 & 0.68 & -0.08 & 0.69 & -0.32 & 0.05 &  &  \\ 
  HMLDev & 0.07 & -0.64 & -0.13 & -0.02 & 0.77 & -0.07 & 0.51 & -0.21 & -0.01 & 0.49 & -0.45 &  \\ 
   \hline
\end{tabular}

	\caption{FactorsCorrelation Matrix (based on full sample)}
	\label{table:correlation}
\end{sidewaystable}

\begin{sidewaystable}[ht]
	\centering
	% latex table generated in R 3.2.3 by xtable 1.8-2 package
% Wed Feb 24 16:09:00 2016
\begin{tabular}{ccc}
  \hline
Model & DF & log margLike \\ 
  \hline
MOM + Mkt.RF + HML + RMW + CMA + QMJ + ME + ROE + HMLDev & 6 & 9713.58 \\ 
  MOM + Mkt.RF + SMB + HML + RMW + CMA + QMJ + ME + HMLDev & 6 & 9713.53 \\ 
  MOM + Mkt.RF + HML + RMW + CMA + QMJ + ME + ROE & 6 & 9713.47 \\ 
  MOM + Mkt.RF + SMB + HML + RMW + CMA + QMJ + ME & 6 & 9712.45 \\ 
  MOM + Mkt.RF + SMB + HML + RMW + CMA + QMJ + ROE + HMLDev & 6 & 9712.26 \\ 
  MOM + Mkt.RF + SMB + HML + RMW + CMA + QMJ + ROE & 6 & 9712.21 \\ 
  constant + MOM + Mkt.RF + HML + RMW + CMA + QMJ + ME + ROE + HMLDev & 6 & 9711.61 \\ 
  MOM + Mkt.RF + SMB + HML + RMW + CMA + QMJ + ME + IA + HMLDev & 6 & 9711.1 \\ 
  MOM + Mkt.RF + HML + RMW + CMA + QMJ + ME + IA + HMLDev & 6 & 9710.89 \\ 
  MOM + Mkt.RF + HML + RMW + CMA + QMJ + ME + HMLDev & 6 & 9710.62 \\ 
  constant + MOM + Mkt.RF + SMB + HML + RMW + CMA + QMJ + ROE + HMLDev & 6 & 9710.29 \\ 
  MOM + Mkt.RF + SMB + HML + RMW + CMA + QMJ + ME + IA & 6 & 9710.11 \\ 
  MOM + Mkt.RF + SMB + HML + RMW + CMA + QMJ + ME + ROE & 6 & 9710.07 \\ 
  MOM + Mkt.RF + HML + RMW + CMA + QMJ + ME + IA + ROE & 6 & 9709.84 \\ 
  constant + MOM + Mkt.RF + HML + RMW + CMA + QMJ + ME + ROE & 6 & 9709.69 \\ 
  MOM + Mkt.RF + HML + RMW + CMA + QMJ + ME + IA & 6 & 9709.68 \\ 
  MOM + Mkt.RF + HML + RMW + CMA + QMJ + ME & 6 & 9709.58 \\ 
  MOM + Mkt.RF + SMB + HML + RMW + CMA + QMJ + ME + ROE + HMLDev & 6 & 9708.89 \\ 
  constant + MOM + Mkt.RF + SMB + HML + RMW + CMA + QMJ + ROE & 6 & 9708.65 \\ 
  MOM + Mkt.RF + HML + RMW + CMA + QMJ + ME + IA + ROE + HMLDev & 6 & 9708.33 \\ 
   \hline
\end{tabular}

	\caption{20 Best Models}
	\label{table:bestModels}
\end{sidewaystable}

Depending on the assets on the left hand side, different models may be selected.
We apply our method to 10 value-weighted industry portfolios available at the Kenneth French's website: Consumer NonDurables(NoDur), Consumer Durables(Dur), Manufacturing(Manuf),  Oil, Gas, and Coal Extraction and Products(Enrgy), Business Equipment(HiTec),   Telephone and Television Transmission(Telcm),   Wholesale, Retail, and Some Services (Shops),
Healthcare, Medical Equipment, and Drugs(Hlth),
Utilities(Utils), Other(Other). 
We choose industry-based portfolios instead of characteristics-based portfolios in order to avoid the potential bias that favors models which include factors similar to the sorting used to construct such portfolios.

The 12 candidate factors include (aside from the constant)\footnote{We thank Lu Zhang for providing us the factors constructed in \cite{hou2014digesting}. Other factors are obtained from authors' web-pages.}:
\begin{itemize}
	\item five factors by \cite{fama1993common} and \cite{fama2015five}: market(Mkt.RF), size(SMB), value(HML), profitability(RMW) and investment(CMA)
	\item three factors proposed by \cite{hou2014digesting}: size(ME), profitability(ROE) and investment(IA)
	\item momentum factor(MOM) as in \cite{carhart1997persistence}
	\item liquidity(LIQv) introduced in \cite{stambaugh2003liquidity}
	\item quality factor(QMJ) as offered by \cite{asness2014quality}
	\item alternative value factor(HMLDev) constructed by \cite{asness2013devil}
\end{itemize}
The full sample ranges from Jan 1968 - Dec 2014 (564 observations). The training sample is Jan 1968 - Dec 1979(144 observation). 

We report the mean returns for the assets and factors in tables \ref{table:avPortfolioReturnsMain} and \ref{table:avFactorReturnsMain}. As can be seen from the correlation matrix \ref{table:correlation}, many factors co-move together. 
Sometimes this correlation arises because of the conceptual similarity of the factors. 
In our sample we have two highly correlated size factors: SMB by \cite{fama2015five} and ME by \cite{hou2014digesting}. Another cluster of factors is related to value and investment activity. 
We include two value factors (HML by \cite{fama2015five} and HMLDev by \cite{asness2013devil}) and two investment factors(CMA by \cite{fama2015five} and IA by \cite{hou2014digesting}). 
Finally, we have three profitability-based factors: RMW by \cite{fama2015five}, ROE by \cite{hou2014digesting} and QMJ by \cite{asness2014quality}.  
The question of joint rather than individual significance is especially relevant for this set up given so many correlated factors. 

\subsection{Results}
In total we explore $2^{13}$ models and consider 6 possible distribution of errors: Student-t with 4, 6, 8, 10 and 12 degrees of freedom and normal. 
As can be seen from the table \ref{table:bestModels}, the suggested number of factors is quite large. 
E.g. the top model features 9 factors. 
Best four models are difficult to distinguish as they are similarly supported by the data. 
However, they have many features in common, in particular, the best models always include four out of the five
\cite{fama2015five} factors (everything except SMB), along with momentum, and
quality minus junk. 
Liquidity factor \cite{stambaugh2003liquidity} is never selected.
Usually only one size factor (either SMB or ME) is included. 
However, multiple factors from other groups may be simultaneously selected.
This is particularly true for profitability-based factors (at least two RMW by \cite{fama2015five} and QMJ by \cite{asness2014quality} are always included). 
The same holds for two value factors: HML by \cite{fama2015five} and HMLDev by \cite{asness2013devil}. 
Out of two investment factors the preference is given to the \cite{fama2015five} CMA. 
Multiple conceptually similar factors may be selected if they reflect different information which makes them jointly useful in explaining the cross-section of returns.

All four best models don't include an intercept. 
Consider the best model and compare it with an alternative specification that includes the same factors and a constant.
The difference between these two models is approximately 0.86 on the $\log_{10}$ scale which can be interpreted as substantial support for the model without a constant by the Jeffrey's scale. 
This finding suggests that we can not reject the hypothesis that the intercept should not be included into the model.


