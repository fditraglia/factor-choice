\documentclass[12pt]{article}
\usepackage{enumerate, hyperref}
\usepackage{amsmath, amssymb}
\usepackage[margin=1.1in]{geometry}
\linespread{1.2}



\begin{document}
\section{Motivational Simulation}
\begin{table}[ht]
	\centering
	\begin{tabular}{ccc}
		\hline
		Model & DF & log margLike \\ 
		\hline
		Mkt.RF + SMB + HML & 4 & 7187.32 \\ 
		Mkt.RF + SMB + HML + CMA & 4 & 7177.05 \\ 
		Mkt.RF + SMB + HML + MSFT & 4 & 7170.99 \\ 
		Mkt.RF + SMB + HML + RMW & 4 & 7170.24 \\ 
		Mkt.RF + SMB + HML & 6 & 7169.59 \\ 
		\hline
	\end{tabular}
	\caption{Simulation Results}
\end{table}
In order to motivate our research we simulate asset returns and demonstrate that the true model is selected. The returns are assumed to be follow the famous Fama-French 3 factor(Mkt.RF, HML and SMB) structure without intercept. The errors follow Student-t distribution with 2.5 degrees of freedom. The simulation is based on posterior means obtained when fitting the model to the real data. Other parameters are the same as in the original sample. The pool of candidate models includes all combinations of Fama-French 5 factors(Mkt.RF, HML, SMB, RMW and CMA), a constant and a non-factor asset - Microsoft stock (MSFT). We assume that the researcher does not know the true distribution. Considered distributions include normal and Student-t with 4, 6, 8, 10 and 12 degrees of freedom. We fit in total $6\times 2^{7} = 768$ model. The simulation is based on the sample range Apr 1986 - Dec 2014 (345 observations). The training sample includes observations Apr 1986 - Dec 1990 (57 observations). \\
The simulation setup is described below:
\begin{enumerate}
	\item Fit the Fama-French 3 factor model without an intercept to 10 value-weighted industry portfolios using the full sample. The errors are assumed to follow Student-t distribution with 4 degrees of freedom. 
	\item Simulate a dataset assuming the true parameters  $\gamma$ and $\Omega^{-1} $ to be equal to the posterior means:
	\begin{itemize}
		\item Simulate errors:
		 \begin{equation*}
		 \boldsymbol{\varepsilon}^s_{t}\sim t_{10,2.5 }\left( 0,\Omega \right)
		 \end{equation*}
		\item Simulate returns using values of Fama-French 3 factors observed in the data:
		\begin{equation*}
		\mathbf{y}_t^s = X_t \boldsymbol{\gamma} + \boldsymbol{\varepsilon}^s_t
		\end{equation*}
	\end{itemize}
	\item Estimate all candidate models and evaluate the likelihood. Run the usual two step estimation procedure using the training sample to construct priors for each model.  
\end{enumerate}


The best model selects the true factors. The distribution  is the closest to the truth. \\


\section{Big Run}
Example of a table for the big run. So far $3\times(2^{13}) = 24576$ models were explored. The distributions include normal and Student-t 4 and 12. For all best models errors follow Student-t distribution with 6 degrees of freedom.
\begin{table}[ht]
	\centering
	\begin{tabular}{ccc}
		\hline
		Model & DF & log margLike \\ 
		\hline
		MOM + Mkt.RF + HML + RMW + CMA + QMJ + ME + ROE & 6 & 9690.77 \\ 
		MOM + Mkt.RF + SMB + HML + RMW + CMA + QMJ + ROE & 6 & 9689.14 \\ 
		MOM + Mkt.RF + SMB + HML + RMW + CMA + QMJ + ME & 6 & 9688.97 \\ 
		MOM + Mkt.RF + HML + RMW + CMA + QMJ + ME + ROE + HMLDev & 6 & 9688.26 \\ 
		MOM + Mkt.RF + SMB + HML + RMW + CMA + QMJ + ME + HMLDev & 6 & 9687.89 \\ 
		\hline
	\end{tabular}
	\caption{Best Big Run Models}
\end{table}
\begin{table}[ht]
	\centering
	\begin{tabular}{ccc}
		\hline
		Model & DF &  log margLike \\ 
		\hline
		MOM + Mkt.RF + HML + RMW + CMA + QMJ + ME + ROE & 4 & 9684.65 \\ 
		MOM + Mkt.RF + SMB + HML + RMW + CMA + QMJ + ME & 4 & 9683.9 \\ 
		MOM + Mkt.RF + SMB + HML + RMW + CMA + QMJ + ME + HMLDev & 4 & 9683.75 \\ 
		MOM + Mkt.RF + SMB + HML + RMW + CMA + QMJ + ROE & 4 & 9683.24 \\ 
		MOM + Mkt.RF + HML + RMW + CMA + QMJ + ME + IA & 4 & 9683.08 \\ 
		\hline
	\end{tabular}
	\caption{Student-t with 4 degrees of freedom}
\end{table}


\begin{table}[ht]
	\centering
	\begin{tabular}{ccc}
		\hline
		Model & DF & log margLike \\ 
		\hline
		MOM + Mkt.RF + HML + RMW + CMA + QMJ + ME + ROE & 8 & 9685.24 \\ 
		MOM + Mkt.RF + SMB + HML + RMW + CMA + QMJ + ROE & 8 & 9684.32 \\ 
		MOM + Mkt.RF + HML + RMW + CMA + QMJ + ME + ROE + HMLDev & 8 & 9683.73 \\ 
		MOM + Mkt.RF + SMB + HML + RMW + CMA + QMJ + ME + HMLDev & 8 & 9683.5 \\ 
		MOM + Mkt.RF + SMB + HML + RMW + CMA + QMJ + ME & 8 & 9682.56 \\ 
		\hline
	\end{tabular}
	\caption{Student-t with 8 degrees of freedom}
\end{table}

\begin{table}[ht]
	\centering
	\begin{tabular}{ccc}
		\hline
		Model & DF & log margLike \\ 
		\hline
		MOM + Mkt.RF + HML + RMW + CMA + QMJ + ME + ROE & 10 & 9677.6 \\ 
		MOM + Mkt.RF + HML + RMW + CMA + QMJ + ME + ROE + HMLDev & 10 & 9676.08 \\ 
		MOM + Mkt.RF + SMB + HML + RMW + CMA + QMJ + ROE & 10 & 9675.6 \\ 
		MOM + Mkt.RF + SMB + HML + RMW + CMA + QMJ + ME + HMLDev & 10 & 9675.02 \\ 
		MOM + Mkt.RF + SMB + HML + RMW + CMA + QMJ + ROE + HMLDev & 10 & 9674.87 \\ 
		\hline
	\end{tabular}
	\caption{Student-t with 10 degrees of freedom}
\end{table}
\begin{table}[ht]
	\centering
	\begin{tabular}{ccc}
		\hline
		Model & DF & log margLike \\ 
		\hline
		MOM + Mkt.RF + HML + RMW + CMA + QMJ + ME + ROE & 12 & 9669.05 \\ 
		MOM + Mkt.RF + HML + RMW + CMA + QMJ + ME + ROE + HMLDev & 12 & 9667.82 \\ 
		MOM + Mkt.RF + SMB + HML + RMW + CMA + QMJ + ROE & 12 & 9667.35 \\ 
		MOM + Mkt.RF + SMB + HML + RMW + CMA + QMJ + ME + HMLDev & 12 & 9667.15 \\ 
		MOM + Mkt.RF + HML + RMW + CMA + QMJ + ME + IA + ROE & 12 & 9666.27 \\ 
		\hline
	\end{tabular}
	\caption{Student-t with 12 degrees of freedom}
\end{table}

\begin{table}[ht]
	\footnotesize
	\centering
	\begin{tabular}{ccc}
		\hline
		Model & DF & log margLike \\ 
		\hline
		MOM + Mkt.RF + SMB + HML + RMW + CMA + QMJ + ME + ROE + HMLDev & $\infty$ & 9517.35 \\ 
		MOM + Mkt.RF + SMB + HML + RMW + CMA + QMJ + ME + HMLDev & $\infty$ & 9515.86 \\ 
		MOM + Mkt.RF + SMB + HML + RMW + CMA + QMJ + ME + IA + HMLDev & $\infty$ & 9515.84 \\ 
		MOM + Mkt.RF + HML + RMW + CMA + QMJ + ME + ROE + HMLDev & $\infty$ & 9514.79 \\ 
		constant + MOM + Mkt.RF + SMB + HML + RMW + CMA + QMJ + ME + ROE + HMLDev & $\infty$ & 9514.64 \\ 
		\hline
	\end{tabular}
	\caption{Gaussian}
\end{table}

\end{document}
