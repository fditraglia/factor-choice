%!TEX root = main.tex

\section{Motivational Example}
In order to motivate our research we simulate asset returns and demonstrate that the true factors are selected as the result of the procedure described above. 
As asset returns we take 10 value-weighted Fama-French industry portfolios. 
The returns are assumed to be follow the famous Fama-French 3 factor structure without intercept (Mkt.RF, HML and SMB). 
The errors follow Student-t distribution with 2.5 degrees of freedom. 
The simulation is based on posterior means obtained when fitting the model to the real data. 
Other parameters are the same as in the original sample.
We assume that the researcher does not know the true distribution. 
Considered distributions include normal and Student-t with 4, 6, 8, 10 and 12 degrees of freedom. 
The pool of candidate models includes all combinations of Fama-French 5 factors(Mkt.RF, HML, SMB, RMW and CMA), a constant and a non-factor asset - Microsoft stock (MSFT). 
We add a Microsoft stock  because of the wide-spread concern that non-factors may be selected if they are correlated with the true factors.
Our results indicate this concern is not supported in the Bayesian framework. 
We fit in total $6\times 2^{7} = 768$ model. 
The simulation is based on the sample range Apr 1986 - Dec 2014 (345 observations). 
The training sample includes observations Apr 1986 - Dec 1990 (57 observations).

The simulation setup is described below:
\begin{enumerate}
	\item Fit the Fama-French 3 factor model without an intercept to 10 value-weighted industry portfolios using the full sample. The errors are assumed to follow Student-t distribution with 4 degrees of freedom. 
	\item Simulate a dataset assuming the true parameters  $\gamma$ and $\Omega^{-1} $ to be equal to the posterior means:
	\begin{itemize}
		\item Simulate errors:
		\begin{equation*}
		\boldsymbol{\varepsilon}^s_{t}\sim t_{10,2.5 }\left( 0,\Omega \right)
		\end{equation*}
		\item Simulate returns using values of Fama-French 3 factors observed in the data:
		\begin{equation*}
		\mathbf{y}_t^s = X_t \boldsymbol{\gamma} + \boldsymbol{\varepsilon}^s_t
		\end{equation*}
	\end{itemize}
	\item Estimate all candidate models and evaluate the likelihood. Run the usual two step estimation procedure using the training sample to construct priors for each model.
\end{enumerate}

\begin{sidewaystable}[ht]
	\centering
	\begin{tabular}{rrrrrrrrrrrr}
		\hline
		& & NoDur & Durbl & Manuf & Enrgy & HiTec & Telcm & Shops & Hlth & Utils & Other \\ 
		\hline
		\multirow{3}{*}{Data} & Apr 1986 - Dec 2014 & 0.0080 & 0.0056 & 0.0078 & 0.0080 & 0.0074 & 0.0062 & 0.0073 & 0.0084 & 0.0061 & 0.0058 \\ 
		& Apr 1986 - Dec 1990 & 0.0099 & -0.0033 & 0.0034 & 0.0092 & -0.0037 & 0.0084 & 0.0036 & 0.0095 & 0.0033 & -0.0020 \\ 
		& Jan 1991 - Dec 2014 & 0.0076 & 0.0073 & 0.0086 & 0.0078 & 0.0096 & 0.0057 & 0.0080 & 0.0082 & 0.0066 & 0.0073 \\ 
		\hline
		\multirow{3}{*}{Simulated} & Apr 1986 - Dec 2014 & 0.0035 & 0.0104 & 0.0063 & 0.0036 & 0.0062 & 0.0089 & 0.0045 & 0.0052 & 0.0043 & 0.0085 \\ 
		& Apr 1986 - Dec 1990 & -0.0016 & 0.0039 & 0.0020 & 0.0005 & 0.0032 & 0.0106 & -0.0016 & 0.0025 & 0.0047 & 0.0036 \\ 
		& Jan 1991 - Dec 2014 & 0.0045 & 0.0117 & 0.0071 & 0.0042 & 0.0067 & 0.0086 & 0.0057 & 0.0057 & 0.0043 & 0.0094 \\ 
		\hline
	\end{tabular}
	\caption{Average Portfolio Returns for Simulated and Real Data}
\end{sidewaystable}

\begin{table}[ht]
	\centering
	\begin{tabular}{rrrrrrr}
		\hline
		& Mkt.RF & SMB & HML & RMW & CMA & MSFT \\ 
		\hline
		 Apr 1986 - Dec 2014 & 0.0062 & 0.0011 & 0.0023 & 0.0036 & 0.0033 & 0.0102 \\ 
		Apr 1986 - Dec 1990 & 0.0028 & -0.0068 & 0.0003 & 0.0047 & 0.0054 & 0.0327 \\ 
		Jan 1991 - Dec 2014 & 0.0069 & 0.0027 & 0.0027 & 0.0034 & 0.0029 & 0.0058 \\ 
		\hline
	\end{tabular}
		\caption{Average Factor Returns}
\end{table}
\begin{table}[ht]
	
	\centering
	\begin{tabular}{ccc}
		\hline
		Model & DF & log margLike \\ 
		\hline
Mkt.RF + SMB + HML & 4 & 7187.32 \\ 
Mkt.RF + SMB + HML + CMA & 4 & 7177.05 \\ 
Mkt.RF + SMB + HML + MSFT & 4 & 7170.99 \\ 
Mkt.RF + SMB + HML + RMW & 4 & 7170.24 \\ 
Mkt.RF + SMB + HML & 6 & 7169.59 \\ 
Mkt.RF + SMB + HML + CMA + MSFT & 4 & 7161.34 \\ 
constant + Mkt.RF + SMB + HML & 4 & 7160.41 \\ 
Mkt.RF + SMB + HML + CMA & 6 & 7159.73 \\ 
constant + Mkt.RF + SMB + HML + CMA & 4 & 7157.97 \\ 
Mkt.RF + SMB + HML + RMW + MSFT & 4 & 7152.92 \\ 
Mkt.RF + SMB + HML & 8 & 7152.24 \\ 
constant + Mkt.RF + SMB + HML + MSFT & 4 & 7151.56 \\ 
Mkt.RF + SMB + HML + RMW & 6 & 7151.19 \\ 
constant + Mkt.RF + SMB + HML & 6 & 7149.93 \\ 
constant + Mkt.RF + SMB + HML + RMW & 4 & 7147.47 \\ 
Mkt.RF + SMB + HML + MSFT & 6 & 7147.06 \\ 
Mkt.RF + SMB + HML + RMW + CMA & 4 & 7146.92 \\ 
Mkt.RF + SMB + HML + CMA + MSFT & 6 & 7146.04 \\ 
constant + Mkt.RF + SMB + HML + CMA + MSFT & 4 & 7145.13 \\ 
constant + Mkt.RF + SMB + HML + MSFT & 6 & 7139.5 \\ 
		\hline
	\end{tabular}
	\caption{Motivational Simulation: 20 Best Models}
\end{table}

The best model selects the true factors. 
Even though the true error distribution (Student-t with 2.5 degrees of freedom) was not considered, the distribution of the best model (Student-t with 4 degrees of freedom) is the closest to the truth. 
As can be seen from the results, the procedure correctly identified the factors. 
Models including non-relevant factors or the non-factor (Microsoft stock) are significantly worse on the log scale.
