%!TEX root = ../main.tex
In this paper we revisit one of the classic questions in empirical finance: which factors in combination are useful for explaining the time-series and cross-section behavior of equity and portfolio returns? 
The contribution of this paper is to consider this question from a Bayesian perspective recognizing that proper evaluation of the worth of a factor has to be in the context of models with and without other factors. 
Answering such a question therefore requires the consideration of all possible subset models, where the subset models are essentially special cases of a seemingly unrelated regression (SUR) model with the same subset of factors on the right-hand side but with different asset-specific factor coefficients, and a jointly distributed vector error with an unknown precision (inverse covariance) matrix. 
In the case of a leading set of 12 factors, along with the intercept which can be present or absent in each possible case, this leads to $2^{13}$ possible SUR models, which along with 6 different assumptions about the error distribution, amounts to the comparison of 49152 SUR models. 
We carefully compare these models with the help of objectively constructed priors (one for each of our models) using a training sample, and with the calculation of marginal likelihoods, computed by the method of \cite{chib1995marginal}. 
Marginal likelihoods are proportional to the posterior probability of each model and have recognized finite sample and asymptotic properties. 
In particular, marginal likelihoods include a penalty for complexity (in other words models with more factors do not necessarily gather greater support) and asymptotically pick either the true model (if it is in the class being considered) or find the model that is closet to the true model (if it is not in the class being considered). 
Our comparison focuses on test assets from the current literature on this topic: a collection of 10 asset portfolios and a collection of 10 equities. Our results show ...
